\newpage

\section{Właściwości}

\subsection{Siarczek galu}
Struktura pasmowa i przerwa energetyczna to są kluczowe parametry dla półprzewodników chalkogenowych. \textit{Chalkogenki} – nieorganiczne związki chemiczne, w których anionami są chalkogeny, tj. siarczki, selenki i telurki. Przykładowymi chalkogenkami są związki $\mathbf{GaS}$, $\mathbf{Ga_{2}S_{3}}$, $\mathbf{GaSe}$, $\mathbf{Ga_{2}Se_{3}}$, które należą do związków III-VI (III: \textbf{In}, \textbf{Ga} i VI: \textbf{S}, \textbf{Se}, \textbf{Te}) grupy chemicznej. Związek $\mathbf{Ga_{2}S_{3}}$ jest ważnym członkiem związków III-VI , który może posiadać najszerszą lukę w zespole. Nieścisłość pomiędzy atomami III i VI grup powoduje, że związek ma różne stechiometrie, zróżnicowane fazy krystaliczne i różne formy sieci.

$\mathbf{GaSe}$ i $\mathbf{GaS}$ mogą krystalizować się w sześciakątnej strukturze warstwowej, ale w tych związkach warstwy różnie układają się w stos. $\mathbf{GaSe}$ posiada prostą przerwę energetyczną wynoszącą około 2 eV. Podczas gdy materiał $\mathbf{GaS}$ staje się przewodnikiem ze skośną przerwą energetyczną o wartości 2.53 eV.

$\mathbf{Ga_{2}Se_{3}}$ może posiadać wadliwą strukturę blendy cynkowej, w której 1/3 miejsc kationowych jest pusta (położenie pustych miejsca jest losowe w siatce krystalicznej) jest to oczywiście zdefektowany półprzewodnik z prostą przerwą energetyczną o wartości 2-2.4eV.

Chalkogenki $\mathbf{GaSe}$, $\mathbf{GaS}$, $\mathbf{Ga_{2}Se_{3}}$ mają wartości przerwy energetycznej poniżej 2.55 eV. które mogą być zakatologowane do materiałów w zakresie widzialnyma, ale nie do zastosowań w zakresie UV jak związek $\mathbf{Ga_{2}S_{3}}$

Siarczek galu występuje w dwóch postaciach:
\begin{itemize}
	\item Siarczek galu(II) - $\mathbf{GaS}$
	\item Siarczek galu(III) - $\mathbf{Ga_{2}S_{3}}$
\end{itemize}
%grupa przestrzenna $\mathbf{P\;6_{3}/mmc}$
$\mathbf{GaS}$ tworzy bezbarwne lub żółte kryształki układu heksagonalnego. Kryształ siarczku galu $\mathbf{(GaS)}$ należy do rodziny półprzewodników warstwowych III-VI. Krystalizuje się w sześciokątnej strukturze o parametrach sieci $a = 0,3578$ i $c = 1,547$ nm. Każda warstwa w strukturze krystalicznej składa się z dwóch atomów galu i dwóch atomów siarki ułożonych w stos wzdłuż osi $c$ z powtarzającą się jednostką $\mathbf{S-Ga-Ga-S}$.
\begin{figure}[H]
	\begin{center}
		\includegraphics[width=0.85\linewidth]{Wlasciwosci/GaS/GaS_vesta.png}
		\caption{Schematyczna reprezentacja struktury krystalicznej $\mathbf{GaS}$ [1]}
	\end{center}
\end{figure}

W kryształach $\mathbf{GaS}$ dominują słabe siły van der Waalsa w oddziaływaniach międzywarstwowych. Silne kowalencyjne siły dominują w oddziaływaniach wewnątrzwarstwowych.
$\mathbf{GaS}$ to półprzewodnik szerokopasmowy, który jest obiecującym materiałem. Skośna przerwa energetyczna wynosi $2.5eV$, a prosta wynosi $2.95eV$. Materiał umożliwia
wytwarzanie niebieskich urządzeń emitujących światło [1].

$\mathbf{Ga_{2}S_{3}}$ jest również półprzewodnikiem zdefektowanym z powodu niedopasowania Ga – III i S - VI. Jedna trzecia miejsc kationowych, czyli pozycji galowych nie jest zajęte. Występują luki kationowe, które znacznie wpływają na właściwości fizyczne tego materiału i jego obszar zastosowań. Może mieć następne struktury krystaliczne: jednoskośna, sześciokątna, kubiczna. Najbardziej stabilną i ogólnie znajdowaną strukturę krystaliczną jest struktura jednoskośna.

\textit{Faza $\alpha'$-$Ga_{2}S_{3}$ i $\alpha$-$Ga_{2}S_{3}$} \\
Najbardziej stabilną fazą związku $\mathbf{Ga_{2}S_{3}}$ jest faza $\alpha'$-$Ga_{2}S_{3}$. Ma jednoskośny układ krystalograficzny. Związek $\alpha'$-$Ga_{2}S_{3}$ ma prostą przerwę energetyczną o wartości 3.05 - 3.30 (w zależności od źródła) eV i skośną przerwę energetyczną o wartości 3.4 eV. Wyhodowane kryształki są jasnożółte lub przezroczyste. Powodem występowania żółtego koloru w tym półprzewodniku, gdzie przerwa energetyczna ma energię większą niż najbardziej energetyczny foton światła widzialnego, jest obecność luk kationowych. Luki w pasmie energetycznym zlokalizowane na poziomie 0.8 - 0.9 eV wyżej od najwyższego punktu pasma walencyjnego. Te luki tworzą pułapki elektronowe. Elektrony które przechodzą z pasma przewodnictwa do tych pułapek emitują fotony o energii 2.15 - 2.25 eV co odpowiada światłu żółtemu.

Faza $\alpha'$-$Ga_{2}S_{3}$ może posiadać dwie grupy przestrzenne. Parametry komórki elementarnej dla tego materiału w zależności od grupy przestrzennej wynoszą:
\begin{itemize}
	\item Grupa przestrzenna $Cc$ - $a=1.111\;nm$, $b=0.640\;nm$, $c=0.702\;nm$, $\beta=121.17^{\circ}$;
	\item Grupa przestrzenna $Bb$ -  $a=1.109\;nm$, $b=0.958\;nm$, $c=0.640\;nm$, $\beta=141.15^{\circ}$;
\end{itemize}

\begin{figure}[H]
	\begin{center}
		\includegraphics[width=1.0\linewidth]{Wlasciwosci/Przekroj_Ga2S3.png}
		\caption{Na lewym rysunku przedstawiono zdefektowaną sieć krystaliczną w płaszczyźnie prostopadłej do osi c dla związku $\alpha'$-$Ga_{2}S_{3}$. Kationowe luki tworzą interfejs. Na prawym rysunku pokazano sposób ułożenia i wiązania atomów. Cztery aniony S w wierzchołkach czworościanu i w środku jeden kation $\mathbf{Ga}$ lub luka.}
	\end{center}
\end{figure}

\begin{figure}[H]
	\begin{minipage}[h]{0.47\linewidth}
		\center{\includegraphics[width=1\linewidth]{Wlasciwosci/Ga2S3/Cc/Ga2S3_a.png}} a) \\
	\end{minipage}
	\hfill
	\begin{minipage}[h]{0.47\linewidth}
		\center{\includegraphics[width=1\linewidth]{Wlasciwosci/Ga2S3/Cc/Ga2S3_b.png}} \\b)
	\end{minipage}
	\vfill
	\begin{minipage}[h]{0.47\linewidth}
		\center{\includegraphics[width=1\linewidth]{Wlasciwosci/Ga2S3/Cc/Ga2S3_c.png}} c) \\
	\end{minipage}q
	\hfill
	\begin{minipage}[h]{0.47\linewidth}
		\center{\includegraphics[width=1\linewidth]{Wlasciwosci/Ga2S3/Cc/Ga2S3_vesta.png}} d) \\
	\end{minipage}
	\caption{$\alpha'$-$\mathbf{Ga_{2}S_{3}}$. Grupa przestrzenna $Cc$.}
	\begin{minipage}[h]{0.47\linewidth}
	\center{\includegraphics[width=1\linewidth]{Wlasciwosci/Ga2S3/Bb/Ga2S3_a.png}} a) \\
	\end{minipage}
	\hfill
	\begin{minipage}[h]{0.47\linewidth}
		\center{\includegraphics[width=1\linewidth]{Wlasciwosci/Ga2S3/Bb/Ga2S3_b.png}} \\b)
	\end{minipage}
	\vfill
	\begin{minipage}[h]{0.47\linewidth}
		\center{\includegraphics[width=1\linewidth]{Wlasciwosci/Ga2S3/Bb/Ga2S3_c.png}} c) \\
	\end{minipage}
	\hfill
	\begin{minipage}[h]{0.47\linewidth}
		\center{\includegraphics[width=1\linewidth]{Wlasciwosci/Ga2S3/Bb/Ga2S3_vesta.png}} d) \\
	\end{minipage}
	\caption{$\alpha'$-$\mathbf{Ga_{2}S_{3}}$. Grupa przestrzenna $Bb$.}
\end{figure}

Grupa przestrzenna dla fazy $\alpha$-$Ga_{2}S_{3}$ jest $P6_1$. Ta faza ma sześciokątny układ krystalograficzny. Parametry komórki elementarnej $a=0.639\;nm$, $c=1.804\;nm$.

\begin{figure}[H]
	\begin{minipage}[h]{0.47\linewidth}
		\center{\includegraphics[width=1\linewidth]{Wlasciwosci/Ga2S3/alfa/Ga2S3_a.png}} a) \\
	\end{minipage}
	\hfill
	\begin{minipage}[h]{0.47\linewidth}
		\center{\includegraphics[width=1\linewidth]{Wlasciwosci/Ga2S3/alfa/Ga2S3_b.png}} \\b)
	\end{minipage}
	\vfill
	\begin{minipage}[h]{0.47\linewidth}
		\center{\includegraphics[width=1\linewidth]{Wlasciwosci/Ga2S3/alfa/Ga2S3_c.png}} c) \\
	\end{minipage}q
	\hfill
	\begin{minipage}[h]{0.47\linewidth}
		\center{\includegraphics[width=1\linewidth]{Wlasciwosci/Ga2S3/alfa/Ga2S3_vesta.png}} d) \\
	\end{minipage}
	\caption{$\alpha$-$\mathbf{Ga_{2}S_{3}}$.}
\end{figure}

\textit{Faza $\gamma$-$Ga_{2}S_{3}$} \\
Faza z grupą przestrzenna $F-43m$. To jest niskotemperaturowa faza. Kryształki $\gamma$-$Ga_{2}S_{3}$ są białego koloru. Przerwa energetyczna wynosi 2.96 eV. Parametry komórki elementarnej: $a=0.517\;nm$.

\begin{figure}[H]
	\begin{minipage}[h]{0.47\linewidth}
		\center{\includegraphics[width=0.8\linewidth]{Wlasciwosci/Ga2S3/gamma/Ga2S3_vesta.png}} a) \\
	\end{minipage}
	\hfill
	\begin{minipage}[h]{0.47\linewidth}
		\center{\includegraphics[width=0.8\linewidth]{Wlasciwosci/Ga2S3/gamma/Ga2S3_a.png}} \\b)
	\end{minipage}
	\caption{$\gamma$-$\mathbf{Ga_{2}S_{3}}$.}
\end{figure}

\textit{Faza $\beta$-$Ga_{2}S_{3}$} \\
Ta faza jest nazywana fazą $\beta$, z sześciokątnym układem krystalograficznym. Grupa przestrzenna $P6_{3}mc$ Parametry komórki elementarnej: $a=0.368\;nm$,  $c=0.602\;nm$. Faza $\beta$ związku $Ga_{2}S_{3}$ ma najmniejszą przerwę energetyczną dla tego materiału 2.48 eV.

\begin{figure}[H]
	\begin{minipage}[h]{0.47\linewidth}
		\center{\includegraphics[width=1\linewidth]{Wlasciwosci/Ga2S3/beta/Ga2S3_a.png}} a) \\
	\end{minipage}
	\hfill
	\begin{minipage}[h]{0.47\linewidth}
		\center{\includegraphics[width=1\linewidth]{Wlasciwosci/Ga2S3/beta/Ga2S3_b.png}} \\b)
	\end{minipage}
	\vfill
	\begin{minipage}[h]{0.47\linewidth}
		\center{\includegraphics[width=1\linewidth]{Wlasciwosci/Ga2S3/beta/Ga2S3_c.png}} c) \\
	\end{minipage}q
	\hfill
	\begin{minipage}[h]{0.47\linewidth}
		\center{\includegraphics[width=1\linewidth]{Wlasciwosci/Ga2S3/beta/Ga2S3_vesta.png}} d) \\
	\end{minipage}
	\caption{$\beta$-$\mathbf{Ga_{2}S_{3}}$.}
\end{figure}

\begin{table}[H]
	\begin{tabular}{|c|c|c|c|c|}
		\hline
		\multicolumn{1}{|l|}{\textbf{Nazwa}} & \textbf{\begin{tabular}[c]{@{}c@{}}Układ \\ krystalograficzny\end{tabular}} & \textbf{\begin{tabular}[c]{@{}c@{}}Grupa\\ przestrzenna\end{tabular}} & \textbf{Typ struktury}                                            & \textbf{\begin{tabular}[c]{@{}c@{}}Parametry sieci\\ krystalicznej\end{tabular}} \\ \hline
		$\alpha$-$\mathbf{Ga_{2}S_{3}}$                              & Sześciokątny                                                                & $P6_{1}$                       & \begin{tabular}[c]{@{}c@{}}Superstruktura \\ wurcytu\end{tabular} & $a=0.639\;nm$,$c=1.804\;nm$                                                                   \\ \hline
		\multirow{2}{*}{$\alpha'$-$\mathbf{Ga_{2}S_{3}}$}            & \multirow{2}{*}{Jednoskośny}                                                & $Cc$                          & \begin{tabular}[c]{@{}c@{}}Superstruktura\\ $\alpha$-$\mathbf{Ga_{2}S_{3}}$\end{tabular}  & 
		\begin{tabular}[c]{@{}c@{}}$a=1.111\;nm$, $b=0.640\;nm$,\\ $c=0.702\;nm$,$\beta=121.17^{\circ}$\end{tabular}                                                                            \\ \cline{3-5} 
		&                                                                             & $Bb$                          & \begin{tabular}[c]{@{}c@{}}Superstruktura\\ $\alpha$-$\mathbf{Ga_{2}S_{3}}$\end{tabular}  & 	
		\begin{tabular}[c]{@{}c@{}}$a=1.109\;nm$, $b=0.958\;nm$,\\ $c=0.640\;nm$, $\beta=141.15^{\circ}$\end{tabular}                                                                            \\ \hline
		$\beta$-$\mathbf{Ga_{2}S_{3}}$                              & Sześciokątny                                                                & $P6_{3}mc$                     & Wurcyt                                                            & $a=0.368\;nm$,                                                                                           $c=0.602\;nm$                                                                              \\ \hline
		$\gamma$-$\mathbf{Ga_{2}S_{3}}$                              & Sześcienny                                                                  & $F$-$43m$                       & Blenda cynkowa                                                    & $a=0.517\;nm$                                                                            \\ \hline
	\end{tabular}
	\caption{Wszystkie fazy krystaliczne $\mathbf{Ga_{2}S_{3}}$}
\end{table}

Główne piki na widmie rentgenowskim dla każdej z faz występują pod tymi samymi kątami, jedynie intensywności mają różne. To oznacza że jeżeli Faza kubiczna $\gamma$-$Ga_{2}S_{3}$ ma strukturę blendy cynkowej $\mathbf{ZnS}$ ze zdefektowaną podsiecią kationową, to i każda z faz też będzie miała zdefektowaną podsieć kationową.

\begin{figure}[H]
	\begin{center}
		\includegraphics[width=1.0\linewidth]{Wlasciwosci/SEM_Ga2S3.png}
		\caption{Na tym rysunku są przedstawione zdjęcia $\alpha'$-$Ga_{2}S_{3}$ za pomocą SEM przy powiększeniu : a) 5000 razy, b) 20000 razy.}
	\end{center}
\end{figure}

\subsection{Zastosowania $\mathbf{Ga_{2}S_{3}}$}

Jednym z najważniejszych problemów materiałów półprzewodnikowych stosowanych w fotoelektronice jest opracowanie związków o stabilnych właściwościach pod wpływem promieniowania UV, promieniowania rentgenowskiego i $\gamma$, a także opracowanie detektorów promieniowania jonizującego w oparciu o te materiały. Strukturalne defekty w materiale wywoływane są pod wpływem promieni UV, promieniowania rentgenowskiego i $\gamma$. Koncentracja i rodzaj tych defektów zależy zarówno od dawki promieniowania, jak i rodzaju materiału. Detektory promieniowania rentgenowskiego i promieniowania UV oparte na tych materiałach muszą spełniać warunek, że koncentracja ich własnych defektów jest znacznie wyższa niż koncentracja defektów wywołanych promieniowaniem. Materiały takie jak $\mathbf{A_{2}^{III}C_{3}^{VI}}$, szczególnie $\mathbf{Ga_{2}S_{3}}$, spełniają to wymaganie. $\frac{1}{3}$ pozycji w podsieci kationowej są nieobsadzone i tworzą luki, czyli defekty kryształu. Koncentracja własnych defektów w tych materiałach wynosi około $10^{22}\;cm^{-3}$. Przewodność elektryczna pojedynczych kryształów $\mathbf{Ga_{2}S_{3}}$ w normalnej temperaturze wynosi około $10^{-12}\;\Omega^{-1}cm^{-1}$. Przewodność elektryczna wzrasta 3 razy z domieszką $\mathbf{Cd}$.

Domieszkowany (Cr i Fe) i niedomieszkowany Ga2S3 może być materiałem luminescencyjnym z emisją fal o długości od bliskiej podczerwieni do UV.

$\mathbf{GaS}$ długi czas zajmował wysoką pozycję wśród materiałów do aplikacji THz. Ma bardzo szerokie okna dla 0.62 – 20 $\mu m$ i od 50 $\mu m$ – do THz. Ale jego warstwowa struktura i wynikające z tego słabe właściwości mechaniczne ograniczają możliwości zastosowania tego materiału. $\mathbf{Ga_{2}S_{3}}$ jest przezroczysty dla 0.44 – 25 $\mu m$ co jest pokazano na wykresie poniżej:

\begin{figure}[H]
	\begin{center}
		\includegraphics[width=0.8\linewidth]{Wlasciwosci/Widmo-absorpcji-Ga2S3.png}
		\caption{Widmo absorpcji dla $Ga_{2}S_{3}$.}
	\end{center}
\end{figure}

Materiał $\mathbf{Ga_{2}S_{3}}$ ma zastosowanie optyczne i optoelektroniczne (diody elektroluminescencyjne (LED), absorber UV w fotowoltaicznych urządzeniach, generator drugiej harmonicznej, generator trzeciej harmonicznej np. w materiale $\mathbf{Ti_{2}S}$-$\mathbf{Ga_{2}S_{3}}$-$\mathbf{GeS_{2}}$), ze względu na swoją skośną i szeroką przerwę energetyczną). (duże czyli bulk kryształy) Struktura cienkowarstwowa Ga2S3/In/Ga2S3 ma potencjalne zastosowanie jako rezonator mikrofalowy. Jego dwójłomność 0,025 jest większa niż $\mathbf{CdSe}$, która pozwala dopasować fazę SHG dla długości fali dłuższej niż 1910 $\mu m$.

Kolejne z możliwych zastosowań tego materiału jest pasywacja powierzchni półprzewodnikowej III - V tj. do utworzenia "natywnej" warstwy siarczkowej na $\mathbf{GaAs}$ lub $\mathbf{InP}$ poprzez siarkowanie na powierzchni; tak, że rekombinacja powierzchni $\mathbf{GaAs}$ lub $\mathbf{InP}$ można radykalnie stłumić, co z kolei znacząco poprawia wydajność urządzenia W dziedzinie nauki o materiałach i technologii brakuje skutecznych warstw pasywacji powierzchniowej  dla $\mathbf{GaAs}$.

Materiały $\mathbf{GaS}$ i $\mathbf{Ga_{2}S_{3}}$ mają doskonałe właściwości dla soczewek optycznych, optycznych amplifikacji do telekomunikacji i produkcji laserowej i są nietoksyczne.

\newpage

\subsection{Widmo ramanowskie i polaryzacyjne}

Widmo ramanowskie dla związku $\alpha$-$\mathbf{Ga_{2}S_{3}}$ ma 7 pików, które odpowiadają przesunięciom: 1) 119, 2) 135, 3) 148, 4) 238, 5) 309, 6) 331, 7) 392 $cm^{-1}$. Znaczące piki takie jak 1), 2) i 3) (Ga-S2 are mainly due to the Ga-S2 scissoring) 
the band at 238 cm-1 is due to the ring out plane bending of (alfa-Ga2S3). The presence of Ga-S symmetric starching is clearly identified with the large intense spectral band at 392.4 cm-1. \textbf{nie wiem co to znaczy}.

\begin{figure}[H]
	\begin{center}
		\includegraphics[width=1.0\linewidth]{Wlasciwosci/Raman-Ga2S3.png}
		\caption{Widmo ramanowskie dla $\alpha$-$Ga_{2}S_{3}$}
	\end{center}
\end{figure}

W zależności od kierunku polaryzacji światła laserowego, które oddziałuje z fononami w materiale intensywności pików będą się zmieniać. Po zaobserwowaniu tych zmian można uzyskać widmo polaryzacyjne. Przykładowe widmo polaryzacyjne znajduje się poniżej:
 
\begin{figure}[H]
	\begin{center}
		\includegraphics[width=0.8\linewidth]{Wlasciwosci/Spectre-polarization-Ga2S3.png}
		\caption{Przykładowe widmo polaryzacyjne dla trzech pików $\mathbf{A_{1g}}$ $\mathbf{E_{2g}^{2}}$ $\mathbf{E_{2g}^{1}}$. Tu intensywność czyli pole powierzchni piku to odległość od środka do punktu, który odpowiada określonemu kątowi. Ten kąt jest kątem między kierunkiem drgania dipola a promieniowaniem pobudzającym. Widać, że tylko pik $\mathbf{A_{1g}}$ znacząco reaguje na zmianę polaryzacji.}
	\end{center}
\end{figure}

Intensywność piku Ramanowskiego można pokazać taką relacją:
\begin{equation}
	I_{s} \sim |e_{i}\mathbf{R}e_{s}|^{2}
\end{equation}
\begin{itemize}
	\item $e_{i}$ i $e_{s}$ - wwektory polaryzacji promieniowania podającego i rozproszonego odpowiednio.
	\item $\mathbf{R}$ - Tensor ramanowski, który zależy od symetrii kryształu i konkretnego drgającego modu.
\end{itemize}

Wektor elektryczny rozproszonego światła Ramana jest powiązany z wektorem elektrycznym światła padającego przez charakterystyczny tensor Ramana. Unikalny tensor Ramana istnieje dla każdego aktywowanego przez Ramana molekularnego trybu wibracyjnego.

Tensor Ramana jest macierzą 3 x 3, która łączy wektor elektryczny (x1, y1, z1) promieniowania wzbudzającego z wektorem elektrycznym (x2, y2, z2) promieniowania rozproszonego podanego przez:

\begin{figure}[H]
	\begin{center}
		\includegraphics[width=0.5\linewidth]{Wlasciwosci/Raman-Tensor.png}
	\end{center}
\end{figure}

Ponieważ $a_{xy} = a_{yx}$, $a_{yz} = a_{zy}$ i $a_{zx} = a_{xz}$, należy określić tylko sześć składowych tensorowych. W tym przypadku x, y i z są prostokątnymi osiami współrzędnych, które są ustalone na cząsteczce, ale są arbitralnie wybierane. Jeśli główne osie tensora Ramana, które są unikalne dla dowolnego danego pasma Ramana, są wybierane jako układ współrzędnych xyz, to sześć niezerowych składników tensora są zredukowane do trzech elementów diagonalnych: $a_{xx}$, $a_{yy}$ i $a_{zz}$. Tak więc, tensor Ramana może zostać ustalony przez określenie $a_{xx}$, $a_{yy}$ i $a_{zz}$ oraz trzech parametrów kątowych, które ustalają główne osie tensorowe w układzie współrzędnych xyz.

Aby zdefiniować system głównych osi (xyz) lokalnego tensora Ramana, arbitralnie wybieramy trzy nieliniowo ułożone atomy $E_1$, $E_2$ i $A$ w cząsteczce będącej przedmiotem zainteresowania w taki sposób, że oś y jest równoległa do linii łączącej atomy $E_1$ i $E_2$, oś x jest równoległa do linii prostopadłej łączącej atom $A$ z osią y, a oś z jest prostopadła do osi y i x. Chociaż zestaw wybranych osi może nie obejmować całego kąta bryłowego 4$\Pi$, zwykle wystarcza do określenia tensora Ramana. Wynika to z faktu, że każda przerwa między kierunkami dwóch kandydujących osi wprowadza błąd, który jest mały w stosunku do błędów w pomiarach eksperymentalnych.


Wyznaczenie tensorów ramanowskich dla związku $\alpha$-$Ga_{2}S_{3}$ i jest celem tej pracy.



