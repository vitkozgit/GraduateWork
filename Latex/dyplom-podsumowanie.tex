\newpage

\section{Podsumowanie}

Dashiell Hammett - Sokół maltański; Poruszyła się niespokojnie na kozetce, a jej oczy wykonały ruch, jakby chciały bezskutecznie uwolnić się od jego wzroku. Wydawała się jakaś mniejsza, bardzo młoda i przygnębiona.Czy policja musi w ogóle o mnie wiedzieć? - spytała. - Wolałabym raczej umrzeć, proszę pana. Nie mogę teraz tego wytłumaczyć, ale czy mógłby pan jakoś tak zrobić, żeby mnie przed nią ochronić, żebym nie musiała odpowiadać na pytania? Nie mogłabym teraz znieść przesłuchania. Wolałabym umrzeć. Potrafi pan? Może - powiedział Spade - ale muszę w takim razie wiedzieć, o co chodzi. Uklękła u jego stóp i uniosła ku niemu twarz nad mocno splecionymi dłońmi, twarz bladą, pełną napięcia i wystraszoną.Nie jestem dobrym człowiekiem - wykrzyknęła. - Jestem zła... gorsza, niż mógłby pan sądzić... ale nie jestem zupełnie zła. Proszę na mnie spojrzeć. Pan wie, że nie jestem taka zupełnie zła, prawda? Pan o tym wie. Och, jestem taka samotna i tak się boję, i jeżeli pan mi nie pomoże, to nie pomoże mi nikt. Wiem, że nie mam prawa prosić pana o zaufanie, skoro sama” panu nie ufam. Chociaż ja panu ufam, tylko nie mogę panu nic powiedzieć. Nie mogę powiedzieć teraz. Potem, jak już będę mogła, powiem. Boję się, proszę pana. Boję się panu zaufać. Zresztą źle się wyraziłam. Ufam panu, ale... zaufałam już Floydowi i... nie mam nikogo, nikogo prócz pana. Tylko pan może mi pomóc. Pan powiedział, że może mi pomóc. Gdybym panu nie uwierzyła, uciekłabym dzisiaj zamiast do pana dzwonić. Czyż klęczałabym tak przed panem, gdyby kto inny mógł mi pomóc? Wiem, że to nieuczciwe z mojej strony. Ale niech pan będzie wielkoduszny, niech pan nie wymaga ode mnie, żebym stawiała sprawę uczciwie. Jest pan silny, zaradny, dzielny. Z pewnością może pan poświęcić dla mnie trochę tej siły, zaradności i męstwa. Niech mi pan pomoże. Niech mi pan pomoże, bo tak bardzo potrzebna mi pomoc i jeśli pan tego nie uczyni, to nie znajdę nikogo, kto to potrafi zrobić, choćbym nie wiem jak szukała. Niech mi pan pomoże. Nie mam prawa prosić pana o pomoc na ślepo, lecz mimo to proszę. Niechaj pan będzie wielkoduszny. Pan może mi pomóc. Niech mi pan pomoże! Spade, który wstrzymywał oddech przez większą część tej przemowy, odetchnął teraz głęboko i rzekł:Niewiele trzeba będzie pani pomocy. Dobra z pani aktorka. Doskonała. Najwięcej podoba mi się wyraz oczu i drżenie głosu, gdy wypowiada pani takie rzeczy, jak: „Niech mi pan pomoże...”