\newpage

\section{Podsumowanie}

W ramach tej pracy dokonano przeglądu bogatej literatury dotyczącej materiału $\mathbf{Ga_{2}S_{3}}$, jego własności i zastosowań. Dokonano analizy różnych struktur krystalograficznych $\mathbf{Ga_{2}S_{3}}$, wykorzystując program VESTA i pliki .cif dla danych struktur krystalograficznych $\mathbf{Ga_{2}S_{3}}$.

W cienkiej warstwie $\mathbf{Ga_{2}S_{3}}$ wyhodowanej na $\mathbf{GaP}$ wyselekcjonowano pojedynczy kryształ do badań ramanowskich.
W widmie ramanowskim kryształku siarczku galu wyróżnione zostały 7 pików: 1 $\rightarrow$ 117$cm^{-1}$, 2 $\rightarrow$ 143$cm^{-1}$, 3 $\rightarrow$ 149$cm^{-1}$, 4 $\rightarrow$ 235$cm^{-1}$, 5 $\rightarrow$ 309$cm^{-1}$, 6 $\rightarrow$ 330$cm^{-1}$, 7 $\rightarrow$ 390$cm^{-1}$. Położenia tych pików zgadzało się z danymi literaturowymi. W dalszej części pracy zostały uzyskane i zbadane widma polaryzacyjne dla materiału $\mathbf{Ga_{2}S_{3}}$ dla dwóch różnych konfiguracji: VV i VH. 

W celu określenia zależności natężenia pików ramanowskich od polaryzacji wiązki padającej i rozproszonej wykonano dopasowania do odpowiednich pików przy wykorzystaniu funkcji Voight'a. Zależności polaryzacyjne dla odpowiednich pików zostały przedstawione na wykresach Arrheniusa. Na podstawie przeprowadzonej analizy teoretycznej z wykorzystaniem tensora ramanowskiego wykazano zgodność otrzymanych wyników eksperymentalnych z modelem teoretycznym dla struktury monoclinic $\mathbf{Ga_{2}S_{3}}$.

Zależności polaryzacyjne i dopasowane krzywe zgadzają się z dużą dokładnością. Uzyskane w ramach dopasowania parametry tensora ramanowskiego, odpowiadające modowi A$^{'}$ są następujące:
\begin{itemize}
	\item parametr $a$ przyjmuje wartość 1;
	\item parametr $b$ przyjmuje wartości w granicach 0.67 - 0.80;
	\item parametr $d$ przyjmuje wartości w granicach 0.10 - 0.12.
\end{itemize}

Uzyskane dopasowania pozwalają z dużym prawdopodobieństwem określić, że mamy do czynienia z fazą jednoskośną $\alpha'$-$\mathbf{Ga_{2}S_{3}}$ o grupie przestrzennej Cc.

Wyniki uzyskane w ramach tej pracy były częściowo prezentowane na konferencji EMRS w formie plakatu[37]. Na podstawie bardziej szczegółowych wyników uzyskanych w ramach tej pracy została przygotowana i zgłoszona publikacja w czasopiśmie \textit{Journal: Materials Science in Semiconductor Processing}[1].

%-------------------------

%Na widmie ramanowskim kryształku siarczku galu wyróżnione zostały 7 pików. Dla każdego piku na widmie ramanowskim zostało uzyskane 36 punktów na widmie polaryzacyjnym. Dla pików 1,4,7 została dopasowana pojedyncza funkcja Voigt'a. Dla pików 2 i 3 oraz 5 i 6 została dopasowana podwójna funkcja Voigt'a. W wyniku dopasowania zostały uzyskane parametry tensorów ramanowskich dla poszczególnych pików. 

%<>Dla każdego piku na widmie ramanowskim zostało uzyskane 36 punktów na widmie polaryzacyjnym, obracając co 5 stopni polaryzacją ($"$półfalówką$"$). Dla pików 1,4,7 została dopasowana pojedyncza funkcja Voigt'a. Dla pików 2 i 3 oraz 5 i 6 została dopasowana podwójna funkcja Voigt'a. Uzyskane w wyniku dopasowania pole pod krzywą piku odpowiada natężeniu danego piku ramanowskiego. Na wykresie Arrheniusa naniesione zostały znormalizowane wartości natężeń pików w funkcji konta polaryzacji<>.


