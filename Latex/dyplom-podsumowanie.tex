\newpage

\section{Podsumowanie}

W ramach tej pracy dokonano przeglądu bogatej literatury  dotyczącej materiału $\mathbf{Ga_{2}S_{3}}$, jego własności i i zastosowań. Dokonano analizy roznych struktur krystalograficznych Ga2S3, wykorzystujac program VESTA i pliki .cif dla danych struktur krystalograficznych Ga2S3.

W cienkiej warstwie Ga2S3 wychodowanej na GaP wyselekcjonowano pojedynczy krysztal do badan ramanowskich
W widmie ramanowskim kryształku siarczku galu wyróżnione zostały 7 pików: 1 $\rightarrow$ 117$cm^{-1}$, 2 $\rightarrow$ 143$cm^{-1}$, 3 $\rightarrow$ 149$cm^{-1}$, 4 $\rightarrow$ 235$cm^{-1}$, 5 $\rightarrow$ 309$cm^{-1}$, 6 $\rightarrow$ 330$cm^{-1}$, 7 $\rightarrow$ 390$cm^{-1}$. Polozenia tych pikow zgadzalo sie z danymi literaturowymi. W dalszej czesci pracy zosaly uzyskane i zbadane widma polaryzacyjne dla materialu Ga2S3 dla dwuch roznych konfuguracji: VV i VH. 

W celu okreslenia zaleznosci natezenia pikow ramanowskich od polaryzacji wiązki padajacej i rozproszonej wykonano dopasowania do odpowiednich pikow przy wykorzystaniu funkcji Voight'a. Zaleznosci polaryzacyjne dla odpowiednich pikow zostaly przedstawione na wykresach Ahreniusa. Na podstawie przeprowadzone analizy teoretycznej z wykorzystaniem tensra ramanowskieg wykazano zgodnosc otrzymanych wynikow eksperymentalnych z modelem teoretycznym dla struktury monoklinic Ga2S3.

Zależności polaryzacyjne i dopasowane krzywe zgadzają się z dużą dokładnością. Uzyskane w ramach dopasowania parametry tensora ramanowskiego, odpowiadające modowi A$^{'}$ są następujące:
\begin{itemize}
	\item parametr $a$ przyjmuje wartość 1, parametr $b$ jest w granicach 0.67 - 0.80, 
	\item parametr $b$ przyjmuje wartości w granicach 0.67 - 0.80.
	\item parametr $d$ przyjmuje wartości w granicach 0.10 - 0.12.
\end{itemize}

Uzyskane dopasowania pozwalają z dużym prawdopodobieństwem określić, że mamy do czynienia z fazą jednoskośną $\alpha'$-Ga2S3 o grupie przestrzennej Cc.

Wyniki uzyskane w ramach tej pracy były czesciowo prezentowane na konferencji MRS w formie plakatu(Odnosnik do konferencji). Na podstawie bardziej szczegolowych wynikow uzyskanych w ramach tej pracy zostala przygotowana i zgloszona publikacja w czaspismie Journal: Materials Science in Semiconductor Processing[odnosnik].

-------------------------

Na widmie ramanowskim kryształku siarczku galu wyróżnione zostały 7 pików. Dla każdego piku na widmie ramanowskim zostało uzyskane 36 punktów na widmie polaryzacyjnym. Dla pików 1,4,7 została dopasowana pojedyncza funkcja Voigt'a. Dla pików 2 i 3 oraz 5 i 6 została dopasowana podwójna funkcja Voigt'a. W wyniku dopasowania zostały uzyskane parametry tensorów ramanowskich dla poszczególnych pików. 

<>Dla każdego piku na widmie ramanowskim zostało uzyskane 36 punktów na widmie polaryzacyjnym, obracając co 5 stopni polaryzacją ($"$półfalówką$"$). Dla pików 1,4,7 została dopasowana pojedyncza funkcja Voigt'a. Dla pików 2 i 3 oraz 5 i 6 została dopasowana podwójna funkcja Voigt'a. Uzyskane w wyniku dopasowania pole pod krzywą piku odpowiada natężeniu danego piku ramanowskiego. Na wykresie Arrheniusa naniesione zostały znormalizowane wartości natężeń pików w funkcji konta polaryzacji<>.


