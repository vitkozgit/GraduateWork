\newpage

\section{Podsumowanie}

W ramach tej pracy zapoznałem się z bogatą literaturą dotyczącą materiału $\mathbf{Ga_{2}S_{3}}$, jego właściwościami i zastosowaniami.

W niniejszej pracy zostały uzyskane i zbadane widma polaryzacyjne materiału $\mathbf{Ga_{2}S_{3}}$ dla dwóch konfiguracji: VV i VH. Na widmie ramanowskim kryształku siarczku galu wyróżnione zostały 7 pików: 1 $\rightarrow$ 117$cm^{-1}$, 2 $\rightarrow$ 143$cm^{-1}$, 3 $\rightarrow$ 149$cm^{-1}$, 4 $\rightarrow$ 235$cm^{-1}$, 5 $\rightarrow$ 309$cm^{-1}$, 6 $\rightarrow$ 330$cm^{-1}$, 7 $\rightarrow$ 390$cm^{-1}$. 

Dla każdego piku na widmie ramanowskim zostało uzyskane 36 punktów na widmie polaryzacyjnym, obracając co 5 stopni polaryzacją ($"$półfalówką$"$). Dla pików 1,4,7 została dopasowana pojedyncza funkcja Voigt'a. Dla pików 2 i 3 oraz 5 i 6 została dopasowana podwójna funkcja Voigt'a. Uzyskane w wyniku dopasowania pole pod krzywą piku odpowiada natężeniu danego piku ramanowskiego. Na wykresie Arrheniusa naniesione zostały znormalizowane wartości natężeń pików w funkcji konta polaryzacji.

Zależności polaryzacyjne i dopasowane krzywe zgadzają się z dużą dokładnością. Parametry tensora ramanowskiego, który odpowiada modu A$^{'}$ są następujące:
\begin{itemize}
	\item parametr $a$ przyjmuje wartość 1, parametr $b$ jest w granicach 0.67 - 0.80, 
	\item parametr $b$ przyjmuje wartości w granicach 0.67 - 0.80.
	\item parametr $d$ przyjmuje wartości w granicach 0.10 - 0.12.
\end{itemize}

Za pomocą programu VESTA zostały uzyskane rysunki struktury krystalicznej różnych faz materiału $\mathbf{Ga_{2}S_{3}}$, które zostały umieszczone w tej pracy.

Wyniki, które zostały uzyskane w ramach tej pracy były prezentowane na konferencji MRS. Także wyniki pracy zostały umieszczone do czasopisma (Jakiego??).

Uzyskane dopasowania pozwalają z dużym prawdopodobieństwem określić, że mamy do czynienie z fazą jednoskośną. W przypadku innych struktur mamy do czynienia z modami o różnej symetrii.


