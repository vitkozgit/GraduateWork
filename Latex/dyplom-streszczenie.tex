Podziękowania

\newpage

\section*{Streszczenie}

Słowa kluczowe: $\mathbf{GaS}$, $\mathbf{Ga_{2}S_{3}}$, cienkie warstwy, spektroskopia ramanowska, widmo polaryzacyjne, tensor ramanowski.

Praca dotyczy badania widm polaryzacyjnych dla materiału $\alpha'$-$\mathbf{Ga_{2}S_{3}}$, który został wyhodowany na fosforku galu $\mathbf{GaP}$. 

Związek $\alpha'$-$\mathbf{Ga_{2}S_{3}}$ należy do klasy materiałów półprzewodnikowych o dużym potencjale aplikacyjnym w obszarach nanoelektroniki, optoelektroniki, odnawialnych źródeł energii, fotoniki czy źródeł promieniowania terahercowego. Jego zdefektowana struktura daje temu materiału własności, które różnią się od właściwości znanego materiału $\mathbf{GaS}$.

Do 2018 roku materiał $Ga_{2}S_{3}$ jeszcze nie został do końca zbadany jeżeli to dotyczy struktury tego materiału. Wiedza na temat struktury tego materiału jest w źródłach często pomieszana i niespójna. Na przykład często napotykane pomieszane oznaczenia fazy $\alpha$ i $\alpha'$. Tą niespójność można wytłumaczyć tym, że widma rentgenowskie i ramanowskie dla każdej z faz $\mathbf{Ga_{2}S_{3}}$ są prawie takie same, jedynie się różnią piki na tych widmach intensywnością, co nie pozwala jednoznacznie stwierdzić czy wyhodowana struktura należy do tejczy innej fazy.

\textcolor{red}{Tutaj 1-3 akapity o uzyskanych wynikach.}

Widma polaryzacyjne siarczku galu (III) uzyskano w laboratorium fizyki\textcolor{red}{???} na Wydziale Fizyki Politechniki Warszawskiej, a częściowe wyniki osiągnięte w ramach tej pracy były prezentowane na konferencjach naukowych.\textcolor{red}{???}  

\newpage

\section*{Abstract}

\newpage

Oświadczenie o samodzielności wykonywania pracy

\newpage

Oświadczenie o udzieleniu Uczelni licencji do pracy

\newpage