Podziękowania

\newpage

\section*{Streszczenie}

Słowa kluczowe: $\mathbf{GaS}$, $\mathbf{Ga_{2}S_{3}}$, cienkie warstwy, spektroskopia ramanowska, widmo polaryzacyjne, tensor ramanowski.

\vspace{10mm}

Praca dotyczy badania widm ramanowskich cienkich warstw $\mathbf{Ga_{2}S_{3}}$ na podłożu fosforku galu (GaP).

Siarczek galu $\mathbf{Ga_{2}S_{3}}$ należy do klasy materiałów półprzewodnikowych o dużym potencjale aplikacyjnym w obszarach nanoelektroniki, optoelektroniki, odnawialnych źródeł energii, fotoniki czy źródeł promieniowania terahercowego. Duża liczba defektów (wakansji galowych) prowadzi do wielu faz krystalicznych Ga2S3 o różnych właściwościach fizycznych, innych niż dla powszechnie znanego siarczku galu $\mathbf{GaS}$.

Do chwili obecnej istnieją kontrowersje odnośnie wyników badań strukturalnych dla różnych faz $\mathbf{Ga_{2}S_{3}}$. Wiedza na temat struktury tego materiału jest w źródłach często pomieszana i niespójna. Tą niespójność można wytłumaczyć tym, że widma rentgenowskie i ramanowskie dla każdej z faz $\mathbf{Ga_{2}S_{3}}$ są prawie takie same. Piki w tych widmach znajdują się prawie w tych samych miejscach a jedyna różnica dotyczy ich względnej intensywności. Nie pozwala to często jednoznacznie określić fazy krystalicznej wyhodowanej struktury Ga2S3. Ten problem pojawia się szczegulnie w przypadku cienkich warst Ga2S3. Celem tej pracy jest określenie fazy krystalicznej wychodowanych cienkich warstw Ga2S3 przy wykorzystaniu pomiarów polaryzacyjnych ramanowskich.

W ramach pracy dokonano obszarnego przeglądu literatury na temat wlasnosci i budowy krystalicznej roznych faz Ga2S3 (rozdzial 1). Rozne fazy krystaliczne zostaly graficznie zilustrowane i omowione. W rozdzale drugim zawarto opis rozpraszania ramanowskiego oraz omówiono technike wykonywaia widm polaryzacyjnych. W rozdziale trzecim opisano technologię wytwarzania warstwy Ga2S3 oraz opisane konfiguracje spektrometru ramanowskiego wykorzystywanego w ramach tej pracy. W rozdziałach 4 i 5 przedstawiono wyniki badań oraz dokonano ich analizy. W ramach pracy zostały zmierzone i przeanalizowane widma polaryzacyjne materiału $\mathbf{Ga_{2}S_{3}}$ dla dwóch konfiguracji: VV i VH.

Na podstawie analizy widm polaryzacyjnych oraz własności tensora ramanowskiego stwierdzono, że badana warstwa Ga2S3 wykazuje strukturę jednoskośną, charakteryzującą się grupą przestrzenną Cc. Faza ta w literaturze jest oznaczana jako $\alpha'$-$\mathbf{Ga_{2}S_{3}}$.

W rozdziale 6 podsumowanie wynikow a w rozdziale 7 spis bibliografii (Mniej wiecej tak skonczyc)

Można dodac zdanie ze te graficzne przedstawienia roznych struktor byly uzyskanr w oparciu plikow .cif i uzylem programu VESTA

Graficzne przedstawienie struktur zostalo wykonane za pomoca programu VESTA w oparciu o pliki .cif dla danych struktur krystalograficznych Ga3S3 (Przy rozdziale pierwszym)

<Do rozdzialu 1 dodac to zdanie>W pracy została opisana każda z czterech faz materiału $\mathbf{Ga_{2}S_{3}}$: heksagonalna faza $\alpha$, jednoskosna faza $\alpha'$, heksagonalna faza $\beta$ i kubiczna faza $\gamma$.<> 

Widma polaryzacyjne siarczku galu (III) uzyskano w laboratorium fizyki\textcolor{red}{???} na Wydziale Fizyki Politechniki Warszawskiej, a częściowe wyniki osiągnięte w ramach tej pracy były prezentowane na konferencjach naukowych.\textcolor{red}{???} 

\newpage

\section*{Abstract}

\newpage

Oświadczenie o samodzielności wykonywania pracy

\newpage

Oświadczenie o udzieleniu Uczelni licencji do pracy

\newpage