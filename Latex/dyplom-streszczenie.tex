\pagestyle{empty}
Podziękowania (Będzie dołączone)

\vspace{10cm}



\newpage

\section*{Streszczenie}

Tytuł pracy: Rozpraszanie ramanowskie w próbkach objętościowych i cienkich warstwach $\mathbf{Ga_{2}S_{3}}$

\vspace{10mm}

Praca dotyczy badania widm ramanowskich cienkich warstw $\mathbf{Ga_{2}S_{3}}$ na podłożu fosforku galu ($\mathbf{GaP}$).

Siarczek galu $\mathbf{Ga_{2}S_{3}}$ należy do klasy materiałów półprzewodnikowych o dużym potencjale aplikacyjnym w obszarach nanoelektroniki, optoelektroniki, odnawialnych źródeł energii, fotoniki czy źródeł promieniowania terahercowego. Duża liczba defektów (wakansji galowych) prowadzi do wielu faz krystalicznych $\mathbf{Ga_{2}S_{3}}$ o różnych właściwościach fizycznych, innych niż dla powszechnie znanego siarczku galu $\mathbf{GaS}$.

Do chwili obecnej istnieją kontrowersje odnośnie wyników badań strukturalnych dla różnych faz $\mathbf{Ga_{2}S_{3}}$. Wiedza na temat struktury tego materiału jest w źródłach często pomieszana i niespójna. Tą niespójność można wytłumaczyć tym, że widma rentgenowskie i ramanowskie dla każdej z faz $\mathbf{Ga_{2}S_{3}}$ są prawie takie same. Piki w tych widmach znajdują się prawie w tych samych miejscach, a jedyna różnica dotyczy ich względnej intensywności. Nie pozwala to często jednoznacznie określić fazy krystalicznej wyhodowanej struktury $\mathbf{Ga_{2}S_{3}}$. Ten problem pojawia się szczególnie w przypadku cienkich warstw $\mathbf{Ga_{2}S_{3}}$. Celem tej pracy jest określenie fazy krystalicznej wyhodowanych cienkich warstw $\mathbf{Ga_{2}S_{3}}$ przy wykorzystaniu pomiarów polaryzacyjnych ramanowskich.

W ramach pracy dokonano obszarnego przeglądu literatury na temat własności i budowy krystalicznej różnych faz $\mathbf{Ga_{2}S_{3}}$. Różne fazy krystaliczne zostały graficznie zilustrowane i omówione. Została opisana każda z czterech faz materiału $\mathbf{Ga_{2}S_{3}}$: heksagonalna faza $\alpha$, jednoskosna faza $\alpha'$, heksagonalna faza $\beta$ i kubiczna faza $\gamma$. Graficzne przedstawienie struktur zostało wykonane za pomocą programu VESTA w oparciu o pliki .cif dla danych struktur krystalograficznych $\mathbf{Ga_{2}S_{3}}$ (rozdział 1). W rozdziale drugim zawarto opis rozpraszania ramanowskiego oraz omówiono technikę wykonywania widm polaryzacyjnych. W rozdziale trzecim opisano technologię wytwarzania warstwy $\mathbf{Ga_{2}S_{3}}$ oraz opisane konfiguracje spektrometru ramanowskiego wykorzystywanego w ramach tej pracy. W rozdziałach 4 i 5 przedstawiono wyniki badań oraz dokonano ich analizy. W rozdziale szóstym zostało umieszczone podsumowanie wyników, a ostatni rozdział tej pracy zawiera bibliografię. W ramach pracy zostały zmierzone i przeanalizowane widma polaryzacyjne materiału $\mathbf{Ga_{2}S_{3}}$ dla dwóch konfiguracji: VV i VH.

Na podstawie analizy widm polaryzacyjnych oraz własności tensora ramanowskiego stwierdzono, że badana warstwa $\mathbf{Ga_{2}S_{3}}$ wykazuje strukturę jednoskośną, charakteryzującą się grupą przestrzenną Cc. Faza ta w literaturze jest oznaczana jako $\alpha'$-$\mathbf{Ga_{2}S_{3}}$.

\vspace{10mm}

\textit{Słowa kluczowe}: \\
\textit{$\mathbf{GaS}$, $\mathbf{Ga_{2}S_{3}}$, cienkie warstwy, spektroskopia ramanowska, widmo polaryzacyjne, tensor ramanowski}.

\vspace{30mm}

(podpis opiekuna naukowego) \textcolor{white}{---------------------------------------------------} (podpis dyplomanta)

\newpage

\section*{Abstract}

Title of the thesis: Raman scattering in volume samples and thin layers $\mathbf{Ga_{2}S_{3}}$.

\vspace{10mm}

The work concerns the Raman spectra of the thin layers $\mathbf{Ga_{2}S_{3}}$ on the basis of gallium phosphide ($\mathbf{GaP}$).

Gallium sulphide $\mathbf{Ga_{2}S_{3}}$ belongs to the class of semiconductor materials with high application potential in the areas of nanoelectronics, optoelectronics, renewable energy sources, photonics and terahertz radiation sources. The large number of defects (gallim vacancy) leads to many crystalline phases $\mathbf{Ga_{2}S_{3}}$ with different physical properties, other than the well-known gallium sulphide $\mathbf{GaS}$.

To the present, there is controversy regarding the results of structural studies for various phases of $\mathbf{Ga_ {2} S_ {3}}$. Knowledge about the structure of this material is often confused and inconsistent in sources. This inconsistency can be explained by the fact that X-ray and Raman spectra for each phase of $\mathbf{Ga_ {2} S_ {3}}$ are almost the same. The peaks in these spectra are found in almost the same places, and the only difference is in their relative intensity. It often does not allow to unambiguously determine the crystal phase of the grown structure $\mathbf{Ga_ {2} S_ {3}}$.
This problem occurs especially with the thin layers $\mathbf{Ga_{2}S_ {3}}$. The aim of this work is to determine the crystalline phase of grown thin layers $\mathbf{Ga_{2}S_{3}}$ using Raman polarization measurements.

As part of the work, there was an extensive review of the literature on the properties and crystalline structure of the various phases. Different crystal phases have been graphically illustrated and discussed. $\mathbf{Ga_{2}S_{3}}$. Each of the four material phases $\mathbf{Ga_ {2} S_ {3}}$ has been described: hexagonal phase $\alpha$, monoclinic phase $\alpha'$, hexagonal phase $\beta$, cubic phase $\gamma$. The graphical representation of structures was made using the VESTA program based on .cif files for given crystallographic structures $\mathbf{Ga_{2}S_{3}}$ (chapter 1). The second chapter contains a description of Raman scattering and discusses the technique of performing polarizing spectra. The third chapter describes the manufacturing technology of the $\mathbf{Ga_{2}S_{3}}$ layer and the described configurations of the Raman spectrometer used in this work. Chapters 4 and 5 present the results of the investigations and analyze them. The sixth chapter contains a summary of results, and the last chapter of this work contains a bibliography. As part of the work, the polarizing spectra of the material $\mathbf{Ga_{2}S_{3}}$ were measured and analyzed for two configurations: VV and VH.

Based on the analysis of polarization spectra and properties of the Raman tensor it was found the investigation layer $\mathbf{Ga_{2}S_{3}}$ has a monoclinic structure characterized by the space group Cc. This phase in the literature is marked as $\alpha'$-$\mathbf{Ga_{2}S_{3}}$.

\vspace{10mm}

\textit{Keywords}: \\
\textit{$\mathbf{GaS}$, $\mathbf{Ga_{2}S_{3}}$, thin layers, Raman spectroscopy, polarizing spectrum, Raman tensor}.

\vspace{30mm}

(podpis opiekuna naukowego) \textcolor{white}{---------------------------------------------------} (podpis dyplomanta)

\newpage

Oświadczenie o samodzielności wykonywania pracy (Będzie dodane na osobnej kartce)

\newpage

Oświadczenie o udzieleniu Uczelni licencji do pracy (Będzie dodane na osobnej kartce)

\newpage