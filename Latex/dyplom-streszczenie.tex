Podziękowania

\newpage

\section*{Streszczenie}

Słowa kluczowe: $\mathbf{GaS}$, $\mathbf{Ga_{2}S_{3}}$, cienkie warstwy, spektroskopia ramanowska, widmo polaryzacyjne, tensor ramanowski.

\vspace{10mm}

Praca dotyczy badania widm polaryzacyjnych dla materiału $\alpha'$-$\mathbf{Ga_{2}S_{3}}$, który został wyhodowany na fosforku galu $\mathbf{GaP}$. 

Związek $\mathbf{Ga_{2}S_{3}}$ należy do klasy materiałów półprzewodnikowych o dużym potencjale aplikacyjnym w obszarach nanoelektroniki, optoelektroniki, odnawialnych źródeł energii, fotoniki czy źródeł promieniowania terahercowego. Jego zdefektowana struktura daje temu materiału własności, które różnią się od właściwości znanego materiału $\mathbf{GaS}$.

Do 2018 roku materiał $\mathbf{Ga_{2}S_{3}}$ jeszcze nie został do końca zbadany jeżeli to dotyczy struktury tego materiału. Wiedza na temat struktury tego materiału jest w źródłach często pomieszana i niespójna. Na przykład często napotykane pomieszane oznaczenia fazy $\alpha$ i $\alpha'$. Tą niespójność można wytłumaczyć tym, że widma rentgenowskie i ramanowskie dla każdej z faz $\mathbf{Ga_{2}S_{3}}$ są prawie takie same, jedynie się różnią piki na tych widmach intensywnością, co nie pozwala jednoznacznie stwierdzić czy wyhodowana struktura należy do tej czy innej fazy.

Widma ramanowskie i widma rentgenowskie są prawie takie same dla każdej z faz $\mathbf{Ga_{2}S_{3}}$. Wobec tego widmo polaryzacyjne dla tego materiału może być bardzo użyteczne dla określenia fazy $\mathbf{Ga_{2}S_{3}}$. W tej pracy zajmuję się badaniem widma polaryzacyjnego dla $\alpha'$-$\mathbf{Ga_{2}S_{3}}$ która ma grupę przestrzenną $Cc$.

W pracy została opisana każda z czterech faz materiału $\mathbf{Ga_{2}S_{3}}$: sześciokątna faza $\alpha$, jednoskośna faza $\alpha'$, sześciokątna faza $\beta$ i sześcienna faza $\gamma$. Za pomocą programu VESTA zostały uzyskane struktury krystaliczne dla poszczególnych faz $\mathbf{Ga_{2}S_{3}}$. Także w tej pracy zostały opisane podstawy teoretyczne rozpraszania ramanowskiego. Została opisana metoda pomiaru rozpraszania ramanowskiego, za pomocą spektrometru RENISZAW InVia.

W niniejszej pracy zostały uzyskane i zbadane widma polaryzacyjne materiału $\mathbf{Ga_{2}S_{3}}$ dla dwóch konfiguracji: VV i VH. Na widmie ramanowskim kryształku siarczku galu wyróżnione zostały 7 pików. Dla każdego piku na widmie ramanowskim zostało uzyskane 36 punktów na widmie polaryzacyjnym. Dla pików 1,4,7 została dopasowana pojedyncza funkcja Voigt'a. Dla pików 2 i 3 oraz 5 i 6 została dopasowana podwójna funkcja Voigt'a. W wyniku dopasowania zostały uzyskane parametry tensorów ramanowskich dla poszczególnych pików. 

Widma polaryzacyjne siarczku galu (III) uzyskano w laboratorium fizyki\textcolor{red}{???} na Wydziale Fizyki Politechniki Warszawskiej, a częściowe wyniki osiągnięte w ramach tej pracy były prezentowane na konferencjach naukowych.\textcolor{red}{???} 

\newpage

\section*{Abstract}

\newpage

Oświadczenie o samodzielności wykonywania pracy

\newpage

Oświadczenie o udzieleniu Uczelni licencji do pracy

\newpage