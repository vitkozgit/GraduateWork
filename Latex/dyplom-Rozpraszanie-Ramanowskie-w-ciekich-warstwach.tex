\newpage

\section{Rozpraszanie Ramanowskie w ciękich warstwach}
\subsection{Fonony w materiale}
\textbf{Fonon} -  kwazicząstka, kwant energii drgań sieci krystalicznej. Są dwa rodzaje fononów:
\begin{itemize}
	\item{Fonony akustyczne. Powstają w wyniku drgań jednego rodzaju atomów.}
	\item{Fonony optyczne. Powstają w wyniku drgań różnego rodzaju atomów.}
\end{itemize}
Podział fononów jest uzależniony od kształtu relacji dyspersji w pobliżu k=0. \\
Fonony akustyczne wykazują zależność:
\begin{equation}
	\lim_{k \to 0} \omega(k) = 0
\end{equation}
natomiast fonony optyczne:
\begin{equation}
\lim_{k \to 0} \omega(k) = const
\end{equation}
\begin{figure}[H]
	\begin{center}
		\includegraphics[width=0.8\linewidth]{Rozpraszanie-Ramanowskie-Ciekie/Phonons.png}
		\caption{Krzywe dyspersyjne dla liniowego łańcucha dwuatomowego.[5]}
	\end{center}
\end{figure}
Dla kryształu zawierającego N(>2) różnych atomów w komórce prymitywnej relacja dyspersji zawiera trzy gałęzie akustyczne oraz $\alpha$N-3 gałęzie optyczne, gdzie $\alpha$ to jest wymiar. Więc dla liniowego łańcucha dwuatomowego N=2 mamy jedną gałąź optyczną i jedną akustyczną. A dla trójwymiarowej komórki prostej składającej się z dwóch różnych cząsteczek będą 3 krzywe optyczne i 3 akustyczne.

Przy rozpraszaniu fotonów na fononach powinny być spełnione dwa prawa zachowania: 

Prawo zachowania energii:
\begin{equation}
	\hbar \mathbf{\omega_{i}} = \hbar \mathbf{\omega_{s}} \pm \hbar \mathbf{\Omega_{fonon}}
\end{equation}
\begin{itemize}
	\item $\omega_{i}$ - częstotliwość fotonu padającego;
	\item $\omega_{s}$ - częstotliwość fotonu rozproszonego;
	\item $\Omega_{fonon}$ - częstotliwość fononu;
	\item $\hbar$ - stała Plancka.
\end{itemize}

Prawo zachowania pędu:
\begin{equation}
	\hbar \mathbf{k_{i}} = \hbar \mathbf{k_{s}} \pm \hbar \mathbf{K_{fonon}}
\end{equation}
\begin{itemize}	
	\item $k_{i}$ - wektor falowy fotonu padającego;
	\item $k_{s}$ - wektor falowy fotonu rozproszonego;
	\item $K_{fonon}$ - wektor falowy fononu;
	\item $\hbar$ - stała Plancka.
\end{itemize}

Pęd fononu jest znacznie większy od pędu fotonu, a energia fotonu jest 
znacznie większa od energii fononu. To znaczy że uczęstniczą w oddziaływaniu tylko te 
fonony co mają mały pęd.

RYSUNEK

Z powyższego rysunku widzimy, że w oddziaływaniach przyjmują udział tylko optyczne fonony co znajdują się w środku strefy Brillouina. Akustyczne fonony nie biorą udziału dlatego, że dla $k \rightarrow 0$ energia też dąży do zera.

\vspace{1cm}
\textbf{ZAPYTAĆ. O obsadzeniu i dlaczego nie liczy się antystoks} \\ 
Prawdopodobieństwo obsadzenia stanu energetycznego fononem jest proporcjonalne do:
\begin{equation}
	\sim \exp^{-\frac{E}{kT}} 
\end{equation}
\begin{itemize}
	\item{$E$ - energia stanu energicznego};
	\item{$k$ - stała Boltzmanna};
	\item{$T$ - temperatura w Kelwinach}.
\end{itemize}
to znaczy że stosunek intensywności promieniowania rozproszonego w widmie antystokesowskim
do intensywności promieniowania pobudzającego jest:
\begin{equation}
	\frac{I_{ants}}{I} \sim \exp^{-\frac{E}{kT}}
\end{equation}
\begin{itemize}
	\item{$I_{ants}$ - intensywność promieniowania w widmie antystokesowskim};
	\item{$I$ - intensywność promieniowania pobudzającego}
\end{itemize}

\subsection{Co można odczytać z widma Ramanowskiego}
Ważną rolę w widmie Ramanowskim odgrywa szerokość połówkowa pików.
Na podstawie informacji o szerokości połówkowej $\Gamma$ można mówić o czasie życia fononów w próbce:
\begin{equation}
	\Gamma \sim \frac{1}{\tau}
\end{equation}

gdzie $\tau$ - czas życia fononu.

RYSUNEK SZEROKOSCI POLOWKOWEJ 
\\
\textbf{ZAPYTAĆ! O FWHM}
\\
Szerokość połówkowa zależy od:
\begin{itemize}
	\item[1]{Rozmiar próbki. Czy materiał jest cienkowarstwowym, czy bulk.}
	\item[2]{Defekty. Fonony rozpraszają się na defektach, co zmniejsza czas ich życia.}
	\item[3]{Rozpraszanie fononów wskutek efektów anharmonicznych}
\end{itemize}
Na podstawie szerokości piku można uzyskać informację o przewodnictwie cieplnym próbki. 
\\
\textbf{ZAPYTAĆ, jak uzyskać TEMPERATURĘ} 
\\
W zależności od ilości i kształtu pików można zbadać jaki to jest materiał, czy badany materiał jest czystym materiałem bez domieszek.

\subsection{Piki ramanowskie w cienkich warstwach}
W materiałach cienkowarstwowych jeden z wymiarów jest rzędu kilku nanometrów. To powoduje że zaczynają odgrywać ważną rolę efekty kwantowe. Z zasady nieoznaczoności Heisenberga:
\begin{equation}
	\Delta p_{fon} \Delta d \geq \frac{\hbar}{2}
\end{equation}
\begin{itemize}
	\item{$\Delta p_{fon}$ - niepewność pomiaru pędu fononu};
	\item{$\Delta d$ - niepewność pomiaru położenia fononu}
\end{itemize}
\textbf{ZAPYTAĆ!jak przejść z tej regóły do kształtu piku}
Pik jest rozmazany w kierunku wyższych energii.
\\
RYSUNEK PIKU ROZMAZANEGO
\\



\subsection{Dane dla pików ramanowskich GaP Ga2S3}

\subsection{przyrząd pomiarowy, rysunek, opis}
